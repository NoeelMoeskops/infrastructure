\addtocounter{section}{1}
\section{Security in Computer Networks}
\subsection{What is Network Security?}
Een veilige verbinden kan bestaan uit de volgende eigenschappen van \textbf{secure communication}
\begin{itemize}
    \item \textit{Confidentiality} Het bericht kan alleen worden gelezen door de ontvangen er en de verzender. Dit kan je krijgen door gebruikt te maken van \textbf{encrypted}
    \item \textit{Message intergrity} Je wilt er zeker van zijn dat het bericht wat je ontvangt ook precies hetzelfde is wat jij hebt verstuurt.
    \item \textit{End-point authentication} Je wilt er zeker van zijn dat de ontvanger ook echt is wie hij zecht dat hij is.
    \item \textit{Operational security} Je wilt dat je netwerk veilig is om op te werken.
\end{itemize}
\subsection{Principles of Cryptography}
Waarneer je blanko tekst versuurt staat dat bekent als \textbf{cleartext/cleartext} je kan met een \textbf{encryption algorithm} je data encrypten, dan staat het bekent als \textbf{ciphertext}
\newline
Om data geencrypt te versturen maak je gebruik van \textbf{keys} Als je data encrypt met een key, de data verstuurt naar een ontvanger en de ontvanger gebruikt dezelfde key om de data te decrypte noem je dit \textbf{symmetric key system}. Het probleem hiervan is dat je eerst een key moet afspreken.
\newline
Met \textbf{public key system} heeft iedere host 2 keys, een public en een private. Je private key deel je met
niemand. Deze gebruik je om berichte te decrypten (of encrypten maar dan komt later). Je public key iedereen bij.
Iemand kan dus een bericht encrypten met jou public key die kan dan alleen worden gedecrypt met jou private key.

\subsubsection{Symmetric Key Cryptography}
\textbf{Monoalphabetic cipher} is een encryptie dat je een letter door een andere letter vervangt.
drie verschillende aanvallen op encryptie:
\begin{itemize}
    \item \textbf{Ciphertext-only attack} Met deze aanval heb je toegang tot de ciphertext zonder identicatie wat het
    bericht zecht.
    \item \textbf{Known-plaintext attack} Nu heb je de cipertext en weet je (gedeeltelijk) wat er in het bericht
    staat, makkelijker te decrypten dus.
    \item \textbf{Chosen-plaintext attack} Nu kan de aanvaller text incrypten opdezelfde methode als het
    oorspronkelijke bericht.
\end{itemize}
\textbf{Polyalpabetic encryption} is dat je op basis van de positie van de zin de letter een andere letter maakt.

\subsubsection{Public Key Encryption}